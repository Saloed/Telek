\include{settings}

\setcounter{tocdepth}{3}

\begin{document}	% начало документа

% Титульная страница
\include{titlepage}

% Содержание
\include{ToC}


\section{Цель работы}
Изучение частотной и фазовой модуляции и демодуляции сигналов.
\section{Постановка задачи}
\begin{enumerate}
\item  сгенерировать однотональный сигнал низкой частоты 
\item  выполнить фазовую модуляцию и демодуляцию 
\item  выполнить частотную модуляцию и демодуляцию 
\item  получить спектр модулированного сигнала
\end{enumerate}

\section{Теоретическая информация}

\subsection{Модуляция}
Модуляция --- это перенос спектра сигналов из низкочастотной области на заданную частоту. 
Это применяется для передачи сигнала в заданном частотном диапазоне.
Для модулирующего (исходного) сигнала $ S(t) $ в канале связи для передачи формируется  вспомогательный периодический высокочастотный сигнал $u(t)=f(t, [a_1,   a_2,   ...   a_m])$. Параметры $a_i$ определяют форму сигнала. 
При модуляции исходный сигнал $S(t)$ переносят на один из параметров $a_i$, форма сигнала $u(t)$ (несущей) изменяется и 
служит для переноса информации, содержащейся в сигнале $S(t)$. Обратная операция выделения сигнала $S(t)$ из 
модулированного сигнала $u(t)$ называется демодуляция.

\subsection{Однотональный сигнал}

Для генерации гармонического сигнала можно воспользоваться формулой\\ $signal = A*cos(2*\pi * f*t + \varphi)$,
 где $ A $ --- амплитуда сигнала, $f$ --- частота, $t$ --- вектор отсчетов времени, $\varphi$ --- смещение по фазе.

\subsection{Угловая модуляция}

При угловой модуляции в несущем гармоническом колебании $u(t) = U_m cos(\omega t + \varphi)$ 
 значение амплитуды колебаний $U_m$ остается постоянным, а информация $s(t)$ переносится либо на частоту $\omega$, 
 либо на фазовый угол $\varphi$. В обоих случаях текущее значение фазового угла гармонического 
 колебания $u(t)$ определяет аргумент $\psi (t) = \omega t + \varphi$ ,
  который называется полной фазой колебания.

\subsubsection{Фазовая модуляция}
При фазовой модуляции модулирующий сигнал определяет фазу несущего колебания
$\phi(t) = k s(t)$. Сигнал с фазовой модуляцией имеет вид 
\begin{equation}
    u(t) = U_m \cos(\omega_0 t + k s(t))
\end{equation}


Изображение сигнала после фазовой модуляции приведено ниже на рис.~\ref{pic:Phase_mod_theor} :
\begin{figure}[H]
	\begin{center}
		\includegraphics[scale=0.7]{Phase_mod_theor}
		\caption{Фазовая модуляция сигнала} 
		\label{pic:Phase_mod_theor} % название для ссылок внутри кода
	\end{center}
\end{figure}


\subsubsection{Частотная модуляция}

При частотной модуляции модулирующий сигнал определяет частоту несущего колебания.
Сигнал с частотной модуляцией имеет вид  
\begin{equation}
	u(t) = U_m cos(\omega_0 t + k \int_{0}^{t} s(t) dt)
\end{equation}
Изображение сигнала после частотной модуляции приведено на рис.~\ref{pic:Freq_mod_theor} :
\begin{figure}[H]
	\begin{center}
		\includegraphics[scale=0.7]{Freq_mod_theor}
		\caption{Частотная модуляция сигнала} 
		\label{pic:Freq_mod_theor} % название для ссылок внутри кода
	\end{center}
\end{figure}



\section{Ход работы}
Код, написанный во время работы приведён в листинге~\ref{code:code_1}. 

\subsection{Генерация однотонального сигнала}
Получим обычный гармонический сигнал  $s(t) = A*cos(2*\pi * f*t + \varphi)$ (рис.~\ref{pic:signal_one_tone}) и его спектр (рис.~\ref{pic:signal_one_tone_spec}).
\begin{figure}[H]
	\begin{center}
		\includegraphics[scale=0.7]{signal}
		\caption{Однотональный сигнал} 
		\label{pic:signal_one_tone} % название для ссылок внутри кода
	\end{center}
\end{figure}
\begin{figure}[H]
	\begin{center}
		\includegraphics[scale=0.7]{signal_spec}
		\caption{Спектр однотонального сигнала} 
		\label{pic:signal_one_tone_spec} % название для ссылок внутри кода
	\end{center}
\end{figure}


\subsection{Фазовая модуляция}
Сигнал после фазовой модуляции приведён на рис.~\ref{pic:phase_mod_sig_carr}. Его спектр показан на рис.~\ref{pic:phase_mod_sig_carr_spec}.
\begin{figure}[H]
	\begin{center}
		\includegraphics[scale=0.7]{mod_sig_p}
		\caption{Фазово-модулированный сигнал} 
		\label{pic:phase_mod_sig_carr} % название для ссылок внутри кода
	\end{center}
\end{figure}
\begin{figure}[H]
	\begin{center}
		\includegraphics[scale=0.7]{mod_sig_p_spec}
		\caption{Спектр фазово-модулированного сигнала} 
		\label{pic:phase_mod_sig_carr_spec} % название для ссылок внутри кода
	\end{center}
\end{figure}

\subsection{Демодуляция фазовой модуляции}
Демодуляция фазовой модуляции представлена на рис.~\ref{pic:phase_demod_sig}, а спектр демодулированного сигнала на рис.~\ref{pic:phase_demod_sig_spec}.
\begin{figure}[H]
	\begin{center}
		\includegraphics[scale=0.7]{demod_sig_p}
		\caption{Фазово-демодулированный сигнал} 
		\label{pic:phase_demod_sig} % название для ссылок внутри кода
	\end{center}
\end{figure}
\begin{figure}[H]
	\begin{center}
		\includegraphics[scale=0.7]{demod_sig_p_spec}
		\caption{Спектр фазово-демодулированного сигнала} 
		\label{pic:phase_demod_sig_spec} % название для ссылок внутри кода
	\end{center}
\end{figure}
Как видно по графикам сигнал после демодуляции совпадает с модулируемым исходным сигналом.

\subsection{Частотная модуляция}

Сигнал после частотной модуляции приведён на рис.~\ref{pic:freq_mod_sig}. Его спектр показан на рис.~\ref{pic:freq_mod_sig_spec}.

\begin{figure}[H]
	\begin{center}
		\includegraphics[scale=0.7]{mod_sig_f}
		\caption{Частотно-модулированный сигнал} 
		\label{pic:freq_mod_sig} % название для ссылок внутри кода
	\end{center}
\end{figure}

\begin{figure}[H]
	\begin{center}
		\includegraphics[scale=0.7]{mod_sig_f_spec}
		\caption{Спектр частотно-модулированного сигнала} 
		\label{pic:freq_mod_sig_spec} % название для ссылок внутри кода
	\end{center}
\end{figure}

\subsection{Демодуляция частотной модуляции}

Демодуляция частотной модуляции представлена на рис.~\ref{pic:freq_demod_sig}, а спектр демодулированного сигнала на рис.~\ref{pic:freq_demod_sig_spec}.
\begin{figure}[H]
	\begin{center}
		\includegraphics[scale=0.7]{demod_sig_f}
		\caption{Частотно-демодулированный сигнал} 
		\label{pic:freq_demod_sig} % название для ссылок внутри кода
	\end{center}
\end{figure}
\begin{figure}[H]
	\begin{center}
		\includegraphics[scale=0.7]{demod_sig_f_spec}
		\caption{Спектр частотно-демодулированного сигнала} 
		\label{pic:freq_demod_sig_spec} % название для ссылок внутри кода
	\end{center}
\end{figure}
Как видно по графикам в сигнале после демодуляции присутствуют незначительные отличия от исходного сигнала.

\section{Выводы}

В данной работе нами были исследованы типы аналоговой модуляции и демодуляции, а именно - фазовая и частотная модуляции и демодуляции. Также были построены спектры этих сигналов. И в случае с фазовой модуляцией и в случае с частотной модуляцией,
сигналы были демодулированы с хорошей точностью, что говорит об эффективности использования таких методов модуляции и демодуляции. 
Такие способы модуляции можно применять для высококачественной передачи.

\section{Листинг}
\lstinputlisting[
	language = Matlab,
	label=code:code_1,
	caption={Код использованный при работе},
]{lab5.m}


\end{document}
