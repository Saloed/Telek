\include{settings}

\begin{document}	% начало документа

% Титульная страница
\include{titlepage}

% Содержание
\include{ToC}


\section{Цель работы}
Изучить воздействие фильтра нижних частот на тестовый сигнал с шумом.

\section{Постановка задачи}
Сгенерировать тестовый гармонический сигнал с шумом, синтезировать ФНЧ, отфильтровать сигнал с шумом. Посмотреть, как ФНЧ влияет на спектр сигнала.

\section{Теоретическая информация}
\subsection{Генерация гармонического сигнала с шумом}
Для генерации гармонического сигнала можно воспользоваться формулой $signal = A*cos(2*\pi * f*t + \varphi)$,
 где $ A $ --- амплитуда сигнала, $f$ --- частота, $t$ --- вектор отсчетов времени, $\varphi$ --- смещение по фазе.

Для добавления шума в исходный сигнал необходимо сложить его с другим сигналом, полученным по аналогичной формуле, но для другой частоты.

\subsection{Фильтр нижних частот}
Любой фильтр работает по принципу умножения сигнала в частотной области на коэффициент, зависящий от частоты.
Фильтр усиливает (или не изменяет) частоты в диапазоне и ослабляет вне его. Так, фильтр нижних частот ослабляет частоты выше заданной границы, умножая их на маленький коэффициент. АЧХ такого фильтра показана на рис.\ref{pic:ach_fnc}:
\begin{figure}[H]
	\begin{center}
		\includegraphics[scale=0.7]{ach_fnc}
		\caption{АЧХ фильтра нижних частот} 
		\label{pic:ach_fnc} % название для ссылок внутри кода
	\end{center}
\end{figure}

Фильтры делятся на БИХ (с бесконечной импульсной характеристикой) и КИХ (с конечной импульсной характеристикой).
Основным свойством БИХ фильтров является то, что их импульсная переходная характеристика имеет бесконечную длину во временной области. У КИХ фильтров гарантируется, что с какого-то момента импульсная характеристика станет равна 0.
Это делает их более устойчивыми, по сравнению с БИХ фильтрами. Самая важная особенность КИХ фильтров заключается в возможности получения точной линейной фазовой характеристики. 

Основным методом расчета коэффициентов является модифицированный алгоритм Ремеза --- (Parks-McClellan algorithm). 
Это косвенный итерационный метод для нахождения оптимальных значений с Чебышевской характеристикой фильтра.
 Особенность метода заключается в минимизации ошибки в полосе затухания и пропускания путем Чебышевской аппроксимации импульсной характеристики.
 
В работе используется КИХ фильтр с равномерно пульсирующей АЧХ (equiriple filter). 

\section{Ход работы}

Ход работы можно разделить на 2 части - генерация зашумлённого сигнала и фильтрация этого сигнала.
Код использованный при исследовании приведён в листинге~\ref{code:code}.

\subsection{Генерация гармонического сигнала с шумом}
Для начала получим обычный гармонический сигнал с частотой 30 Гц. Сгенерированный сигнал представлен на рисунке \ref{pic:signal1}:
\begin{figure}[H]
	\begin{center}
		\includegraphics[scale=0.7]{signal1}
		\caption{Гармонический сигнал} 
		\label{pic:signal1} % название для ссылок внутри кода
	\end{center}
\end{figure}

Затем сгенерируем еще один сигнал с более высокой частотой, и прибавим его к имеющемуся. Результат добавления шума в сигнал показан на рисунке \ref{pic:signal2}:
\begin{figure}[H]
	\begin{center}
		\includegraphics[scale=0.7]{signal2}
		\caption{Гармонический сигнал с шумом} 
		\label{pic:signal2} % название для ссылок внутри кода
	\end{center}
\end{figure}

Далее получим спектр сигнала с помощью преобразования Фурье. Спектр гармонического сигнала с шумом приведен на рисунке \ref{pic:signal2_fft}:
\begin{figure}[H]
	\begin{center}
		\includegraphics[scale=0.7]{signal2_fft}
		\caption{Спектр зашумленной гармоники} 
		\label{pic:signal2_fft} % название для ссылок внутри кода
	\end{center}
\end{figure}
Видно, что в сигнале присутствуют 2 гармоники разной частоты.

\subsection{Фильтрация сигнала}

Для фильтрации будем использовать КИХ фильтр низких частот с равномерно пульсирующей АЧХ. 
Коэффициенты фильтра получены с помощью функции Matlab (листинг~\ref{code:code}, строка 21)

Отфильтрованный полученным фильтром сигнал можно увидеть на рисунке \ref{pic:filter_signal}:
\begin{figure}[H]
	\begin{center}
		\includegraphics[scale=0.7]{filter_signal}
		\caption{Сигнал после прохождения фильтра} 
		\label{pic:filter_signal} % название для ссылок внутри кода
	\end{center}
\end{figure}
Максимальная амплитуда немного уменьшена из-за коэффициента ослабления фильтра, и сигнал устанавливается с небольшой задержкой.

Спектр данного сигнала, полученный с помощью преобразования Фурье, приведен на рисунке \ref{pic:filter_signal_fft}:
\begin{figure}[H]
	\begin{center}
		\includegraphics[scale=0.7]{filter_signal_fft}
		\caption{Спектр отфильтрованного сигнала} 
		\label{pic:filter_signal_fft} % название для ссылок внутри кода
	\end{center}
\end{figure}
На рисунке видна одна гармоника, т.е. фильтр верно отсек гармонику шума, внесенного нами в сигнал.

\section{Выводы}

Нами исследовано прохождение сигнала через линейную цепь фильтра нижних частот.
Фильтрация зашумлённого сигнала --- это свёртка с окном чистой АЧХ.
Идеальное окно имеет вид прямоугольника, но получить его невозможно. 
На практике используется не идеальная аппроксимация, с неполным подавлением 
шума на частотах, близких к частоте среза. 
Это объясняется тем, что аппроксимация имеет неидеальный наклон кривой после частоты среза.

\section{Листинг}
\lstinputlisting[
	label=code:code,
	caption={Код для исследования фильтра},
]{Code.m}
\parindent=1cm

\lstinputlisting[
	label=code:spec,
	caption={Код для получения спектра сигнала},
]{spectrum.m}
\parindent=1cm
\end{document}
